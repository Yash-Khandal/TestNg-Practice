\documentclass[a4paper,12pt]{article}
\usepackage[utf8]{inputenc}
\usepackage[T1]{fontenc}
\usepackage{lmodern}
\usepackage{geometry}
\geometry{margin=1in}
\usepackage{listings}
\usepackage{xcolor}
\usepackage{booktabs}
\usepackage{hyperref}
\hypersetup{colorlinks=true, linkcolor=blue, urlcolor=blue}

\lstset{
    basicstyle=\ttfamily\small,
    keywordstyle=\color{blue}\bfseries,
    stringstyle=\color{red},
    commentstyle=\color{green!60!black},
    numbers=left,
    numberstyle=\tiny,
    stepnumber=1,
    numbersep=5pt,
    showspaces=false,
    showstringspaces=false,
    frame=single,
    breaklines=true,
    breakatwhitespace=true,
    tabsize=4
}

\title{Selenium with Java: Comprehensive Notes (Basic to Advanced)}
\author{Grok}
\date{September 17, 2025}

\begin{document}

\maketitle
\tableofcontents
\newpage

\section{Introduction to Selenium}

\subsection{What is Selenium?}
Selenium is an open-source tool for automating web browsers, ideal for testing web applications like login flows on sites such as Amazon.

\begin{itemize}
    \item \textbf{Why Selenium?} Free, supports multiple browsers (Chrome, Firefox), and languages (Java is robust for automation).
    \item \textbf{Components}:
    \begin{itemize}
        \item \textbf{WebDriver}: Main tool for scripting (our focus).
        \item \textbf{IDE}: Browser extension for recording simple tests.
        \item \textbf{Grid}: Runs tests on multiple machines/browsers.
        \item \textbf{RC}: Deprecated, avoid.
    \end{itemize}
\end{itemize}

\subsection{Why Selenium with Java?}
Java’s object-oriented nature, exception handling, and libraries like TestNG make it ideal. Example: Automate a Google search.

\section{Setting Up the Environment}

\subsection{Step 1: Install Java JDK}
\begin{itemize}
    \item Download JDK 8 or 11 from Oracle or OpenJDK.
    \item Verify: \texttt{java -version}.
    \item Set \texttt{JAVA\_HOME} environment variable.
\end{itemize}

\subsection{Step 2: Install IDE}
Use Eclipse (Java EE) or IntelliJ IDEA Community. Create a Maven project for easy dependency management.

\subsection{Step 3: Add Selenium}
\begin{itemize}
    \item Manual: Download Selenium WebDriver JARs from \url{https://selenium.dev}.
    \item Maven: Add to \texttt{pom.xml}:
\end{itemize}

\begin{lstlisting}[language=XML]
<dependency>
    <groupId>org.seleniumhq.selenium</groupId>
    <artifactId>selenium-java</artifactId>
    <version>4.15.0</version>
</dependency>
\end{lstlisting}

For Chrome, download ChromeDriver and add to PATH.

\subsection{Quick Test}
\begin{lstlisting}[language=Java]
import org.openqa.selenium.WebDriver;
import org.openqa.selenium.chrome.ChromeDriver;

public class FirstTest {
    public static void main(String[] args) {
        WebDriver driver = new ChromeDriver();
        driver.get("https://www.google.com");
        System.out.println("Title: " + driver.getTitle());
        driver.quit();
    }
}
\end{lstlisting}

\section{Basic WebDriver Concepts}

\subsection{Launching Browsers}
\begin{itemize}
    \item Chrome: As above.
    \item Firefox:
    \begin{lstlisting}[language=Java]
import org.openqa.selenium.firefox.FirefoxDriver;
WebDriver driver = new FirefoxDriver();
    \end{lstlisting}
    \item Headless:
    \begin{lstlisting}[language=Java]
import org.openqa.selenium.chrome.ChromeOptions;
ChromeOptions options = new ChromeOptions();
options.addArguments("--headless");
WebDriver driver = new ChromeDriver(options);
    \end{lstlisting}
\end{itemize}

\subsection{Navigating}
\begin{itemize}
    \item \texttt{driver.get("url")}: Open page.
    \item \texttt{driver.navigate().to("url")}: Supports back/forward.
    \item \texttt{driver.getTitle()}: Page title.
\end{itemize}

Example: Navigate Amazon.
\begin{lstlisting}[language=Java]
driver.get("https://www.amazon.com");
driver.navigate().to("https://www.google.com");
driver.navigate().back();
System.out.println(driver.getCurrentUrl());
driver.quit();
\end{lstlisting}

\section{Finding Elements (Locators)}

Use \texttt{driver.findElement(By.locator)}. Common locators:

\begin{table}[h]
\centering
\begin{tabular}{l l l l}
\toprule
\textbf{Locator} & \textbf{Description} & \textbf{When to Use} & \textbf{Example} \\
\midrule
By.id & Fast, unique ID & First choice & \texttt{By.id("username")} \\
By.name & Name attribute & Forms & \texttt{By.name("user")} \\
By.className & Class attribute & Styling & \texttt{By.className("form")} \\
By.cssSelector & CSS path & Complex & \texttt{input.form} \\
By.xpath & XML path & Dynamic & \texttt{//input[@id='username']} \\
\bottomrule
\end{tabular}
\caption{Common Selenium Locators}
\end{table}

\subsection{Finding Multiple Elements}
\begin{lstlisting}[language=Java]
List<WebElement> links = driver.findElements(By.tagName("a"));
System.out.println("Total links: " + links.size());
\end{lstlisting}

\subsection{Example: Login Form}
\begin{lstlisting}[language=Java]
WebElement emailField = driver.findElement(By.id("email"));
emailField.sendKeys("test@gmail.com");
WebElement loginBtn = driver.findElement(By.id("loginBtn"));
loginBtn.click();
\end{lstlisting}

\section{Interacting with Elements}

\begin{itemize}
    \item \texttt{sendKeys("text")}: Type.
    \item \texttt{click()}: Click.
    \item \texttt{getText()}: Get text.
    \item \texttt{isDisplayed()}: Check visibility.
\end{itemize}

Example:
\begin{lstlisting}[language=Java]
WebElement nameField = driver.findElement(By.name("fullname"));
nameField.clear();
nameField.sendKeys("John Doe");
driver.findElement(By.id("submit")).click();
\end{lstlisting}

\section{Waits (Handling Dynamic Pages)}

\subsection{Implicit Wait}
\begin{lstlisting}[language=Java]
driver.manage().timeouts().implicitlyWait(Duration.ofSeconds(10));
\end{lstlisting}

\subsection{Explicit Wait}
\begin{lstlisting}[language=Java]
import org.openqa.selenium.support.ui.WebDriverWait;
import org.openqa.selenium.support.ui.ExpectedConditions;

WebDriverWait wait = new WebDriverWait(driver, Duration.ofSeconds(10));
WebElement elem = wait.until(ExpectedConditions.visibilityOfElementLocated(By.id("myButton")));
elem.click();
\end{lstlisting}

\subsection{Fluent Wait}
\begin{lstlisting}[language=Java]
import org.openqa.selenium.support.ui.FluentWait;

Wait<WebDriver> fluentWait = new FluentWait<WebDriver>(driver)
    .withTimeout(Duration.ofSeconds(10))
    .pollingEvery(Duration.ofSeconds(2))
    .ignoring(NoSuchElementException.class);
\end{lstlisting}

\section{Handling Alerts, Frames, Windows}

\subsection{Alerts}
\begin{lstlisting}[language=Java]
Alert alert = driver.switchTo().alert();
alert.accept();
\end{lstlisting}

\subsection{Frames}
\begin{lstlisting}[language=Java]
driver.switchTo().frame("player");
driver.switchTo().defaultContent();
\end{lstlisting}

\subsection{Windows}
\begin{lstlisting}[language=Java]
String parent = driver.getWindowHandle();
Set<String> handles = driver.getWindowHandles();
for(String win : handles) {
    if(!win.equals(parent)) driver.switchTo().window(win);
}
\end{lstlisting}

\section{Mouse and Keyboard Actions}

\subsection{Mouse}
\begin{lstlisting}[language=Java]
import org.openqa.selenium.interactions.Actions;

Actions actions = new Actions(driver);
actions.moveToElement(menu).perform();
\end{lstlisting}

\subsection{Keyboard}
\begin{lstlisting}[language=Java]
actions.sendKeys(Keys.ENTER).perform();
\end{lstlisting}

\section{Taking Screenshots}

\begin{lstlisting}[language=Java]
import org.apache.commons.io.FileUtils;

File screenshot = ((TakesScreenshot) driver).getScreenshotAs(OutputType.FILE);
FileUtils.copyFile(screenshot, new File("error.png"));
\end{lstlisting}

\section{Data Handling}

\subsection{Properties File}
\begin{lstlisting}[language=Java]
Properties prop = new Properties();
prop.load(new FileInputStream("config.properties"));
String url = prop.getProperty("url");
driver.get(url);
\end{lstlisting}

\subsection{Excel (Apache POI)}
\begin{lstlisting}[language=Java]
import org.apache.poi.ss.usermodel.*;

FileInputStream fis = new FileInputStream("data.xlsx");
Workbook wb = WorkbookFactory.create(fis);
String username = wb.getSheet("Sheet1").getRow(0).getCell(0).getStringCellValue();
\end{lstlisting}

\section{Page Object Model (POM)}

\subsection{Base Class}
\begin{lstlisting}[language=Java]
public class BasePage {
    protected WebDriver driver;
    protected WebDriverWait wait;

    public BasePage(WebDriver driver) {
        this.driver = driver;
        this.wait = new WebDriverWait(driver, Duration.ofSeconds(10));
    }
}
\end{lstlisting}

\subsection{Page Class}
\begin{lstlisting}[language=Java]
import org.openqa.selenium.support.FindBy;

public class LoginPage extends BasePage {
    @FindBy(id = "username") private WebElement userField;
    @FindBy(id = "password") private WebElement passField;

    public LoginPage(WebDriver driver) {
        super(driver);
        PageFactory.initElements(driver, this);
    }

    public void login(String user, String pass) {
        userField.sendKeys(user);
        passField.sendKeys(pass);
    }
}
\end{lstlisting}

\section{TestNG Framework}

\subsection{Setup}
Add to \texttt{pom.xml}:
\begin{lstlisting}[language=XML]
<dependency>
    <groupId>org.testng</groupId>
    <artifactId>testng</artifactId>
    <version>7.8.0</version>
</dependency>
\end{lstlisting}

\subsection{Test Example}
\begin{lstlisting}[language=Java]
import org.testng.annotations.*;

public class GoogleTest {
    WebDriver driver;

    @BeforeMethod
    public void setup() {
        driver = new ChromeDriver();
    }

    @Test
    public void searchTest() {
        driver.get("https://google.com");
        driver.findElement(By.name("q")).sendKeys("Selenium");
        Assert.assertTrue(driver.getTitle().contains("Selenium"));
    }

    @AfterMethod
    public void teardown() {
        driver.quit();
    }
}
\end{lstlisting}

\subsection{Data Provider}
\begin{lstlisting}[language=Java]
@DataProvider(name = "loginData")
public Object[][] getData() {
    return new Object[][] {
        {"user1", "pass1"},
        {"user2", "pass2"}
    };
}

@Test(dataProvider = "loginData")
public void loginTest(String user, String pass) {
    System.out.println("Testing with " + user);
}
\end{lstlisting}

\section{Final Tips}

\begin{itemize}
    \item \textbf{Practice}: Automate sites like Facebook login.
    \item \textbf{Errors}: Check locators, waits, PATH.
    \item \textbf{Next}: Cucumber, Jenkins, Appium.
\end{itemize}

\end{document}